%% Adaptado de 
%% http://www.ctan.org/tex-archive/macros/latex/contrib/IEEEtran/
%% Traduzido para o congresso de IC da USP
%%*****************************************************************************
% Não modificar

\documentclass[twoside,conference,a4paper]{IEEEtran}

%******************************************************************************
% Não modificar
\usepackage{IEEEtsup} % Definições complementares e modificações.
\usepackage[latin1,utf8]{inputenc} % Disponibiliza acentos.
\usepackage[english,brazil]{babel}
%\usepackage[brazil,brazilian]{babel}
%% Disponibiliza Inglês e Português do Brasil.
\usepackage{latexsym,amsfonts,amssymb} % Disponibiliza fontes adicionais.
\usepackage{theorem} 
\usepackage[cmex10]{amsmath} % Pacote matemático básico 
\usepackage{url} 
%\usepackage[portuges,brazil,english]{babel}
\usepackage{graphicx}
\usepackage{amsmath}
\usepackage{amssymb}
\usepackage{color}
\usepackage[pagebackref=true,breaklinks=true,letterpaper=true,colorlinks,bookmarks=false]{hyperref}
\usepackage[tight,footnotesize]{subfigure} 
\usepackage[noadjust]{cite} % Disponibiliza melhorias em citações.

\usepackage{listings}
\usepackage{xcolor}

\definecolor{codegreen}{rgb}{0,0.6,0}
\definecolor{codegray}{rgb}{0.5,0.5,0.5}
\definecolor{codepurple}{rgb}{0.58,0,0.82}
\definecolor{backcolour}{rgb}{255,255,255}

\lstdefinestyle{mystyle}{
    backgroundcolor=\color{backcolour},   
    commentstyle=\color{codegreen},
    keywordstyle=\color{magenta},
    numberstyle=\tiny\color{codegray},
    stringstyle=\color{codepurple},
    basicstyle=\ttfamily\footnotesize,
    breakatwhitespace=false,         
    breaklines=true,                 
    captionpos=b,                    
    keepspaces=true,                 
    numbers=left,                    
    numbersep=5pt,                  
    showspaces=false,                
    showstringspaces=false,
    showtabs=false,                  
    tabsize=2
}

\lstset{style=mystyle}
%%*****************************************************************************

\begin{document}
\selectlanguage{brazil}
\renewcommand{\IEEEkeywordsname}{Palavras-chave}

%%*****************************************************************************

\urlstyle{tt}
% Indicar o nome do autor e o curso/nível (grad-mestrado-doutorado-especial)
\title{Project 3 - XX - AI}
\author{%
 \IEEEauthorblockN{Christian Maekawa\,\IEEEauthorrefmark{1}}
 \IEEEauthorblockA{\IEEEauthorrefmark{1}%
                   RA: XXX \\
                   E-mail: XXX@ic.unicamp.br}
 \IEEEauthorblockN{Giovane de Morais\,\IEEEauthorrefmark{1}}
 \IEEEauthorblockA{\IEEEauthorrefmark{1}%
                   RA: XXX \\
                   E-mail: g192683@dac.unicamp.br}
 \IEEEauthorblockN{Maísa Silva\,\IEEEauthorrefmark{1}}
 \IEEEauthorblockA{\IEEEauthorrefmark{1}%
                   RA: XXX \\
                   E-mail: XXX@ic.unicamp.br}
 \IEEEauthorblockN{Matteus Vargas\,\IEEEauthorrefmark{1}}
 \IEEEauthorblockA{\IEEEauthorrefmark{1}%
                   RA: XXX \\
                   E-mail: XXX@ic.unicamp.br}
 \IEEEauthorblockN{Stéfani Fernandes\,\IEEEauthorrefmark{1}}
 \IEEEauthorblockA{\IEEEauthorrefmark{1}%
                   RA: 147939 \\
                   E-mail: s147939@ic.unicamp.br}
}

%%*****************************************************************************

\maketitle

%%*****************************************************************************
% Resumo do trabalho
\begin{abstract}

\end{abstract}

% Indique três palavras-chave que descrevem o trabalho
\begin{IEEEkeywords}
 Palavras-chave
\end{IEEEkeywords}

%%*****************************************************************************
% Modifique as seções de acordo com o seu projeto

\section{Introdução}

%O presente relatório apresenta a descrição das estratégias adotadas para a resolução do Projeto 3 da disciplina Introdução à Inteligência Artificial, ministrada no primeiro semestre de 2020 pela professora Dra. Esther Luna Colombini.
Aqui deve ser apresentada a problemática que iremos abordar, porque ela é relevante e os conceitos que iremos abranger.

\section{Referencial teórico}
Nesta seção, serão apresentados os conceitos e ferramentas utilizados neste trabalho, através de uma revisão da literatura a fim assegurar a compreensão por completo do projeto. 

\subsection{CAPTCHA}

\subsection{Aprendizado de Máquina}
Atualmente a produção de dados está em alto crescimento e vem de diversas fontes como imagens, textos, sons, entre outras fontes. Não são apenas pessoas que os produzem, dispositivos eletrônicos e aplicações também os produzem. 

Apesar das pessoas conseguirem interpretar dados e gerar conhecimento, é impossível trabalhar com uma gama gigantesca. Portanto, é necessário terceirizar essa função para sistemas automatizados que conseguem aprender com os dados e suas mudanças, sendo deveras adaptativo.
Esses sistemas utilizam de aprendizado de máquina que, por sua vez, utiliza-se de ferramentas e tecnologia que buscam responder perguntas e
gerar insights através do consumo dados \cite{santos2018identificaccao}.

A primeira etapa disso é o treinamento, onde há um consumo de dados para criação e parametrização de um modelo de predição, que será usado para gerar previsões de resultados futuros para novos dados \cite{russell2002artificial}. .  

Contudo, para que o próprio treinamento seja satisfatório, os dados devem ser tratados para selecionar quais são importantes de fato, o que requer muito esforço \cite{russell2002artificial}.
Muitas empresas buscam utilizar de Aprendizado de Máquina para suas aplicações a fim de personalizar os seus produtos de forma automatizada para cada usuário, como em recomendação de vídeos, identificação fácil, sistemas de saúde, segurança, entre outras aplicações. Há três grandes categorias de aprendizado de máquina: supervisionado, não-supervisionado e reforço \cite{russell2002artificial}. Aqui, foi usado o aprendizado supervisionado, portanto esse será o foco.
\subsubsection{Aprendizado Supervisionado}

Neste tipo de aprendizado, os dados são rotulados, então eles são corretos para realizar o treinamento do modelo. Esse método é deveras eficiente, já que o sistema pode trabalhar com informações corretas \cite{russell2002artificial}. 

A rotulação dos dados é o cerne deste tipo de aprendizado, pois quando os rótulos são contínuos, então o problema é de regressão e quando forem discretos, então o problema é de classificação \cite{russell2002artificial}.

Vejamos dois exemplos:
\begin{enumerate}
    \item Regressão — Dada uma imagem de homem/mulher, tentar prever sua idade com base em dados da imagem.
    \item Classificação — Dada um exemplo de tumor cancerígeno, tentar prever se ele é benigno ou maligno através do seu tamanho e idade do paciente.
\end{enumerate}

Em resumo, o aprendizado supervisionado realiza um treinamento de um modelo por meio de um histórico de dados rotulados e realiza predições de outros dados futuros não rotulados \cite{santos2018identificaccao}, onde no primeiro exemplo seria a imagem de homem/mulher e no segundo exemplo as características de um tumor.

%\subsection{Redes Neurais Artificiais (RNA)}
\subsection{Redes Neurais Convolucionais (CNN)}
\subsubsection{Arquitetura}
\subsubsection{Camada Convolucional}
\subsubsection{Camada de Pooling}
\subsubsection{Camada Totalmente Conectada}
\subsection{Funções de Ativação}
\subsubsection{Função Sigmoide}
\subsubsection{Função Unidade Linear Retificada (ReLU)}
\subsubsection{Função SoftMax}
\subsubsection{Função de Perda Cross-Entropy}
\subsubsection{Inicialização de Pesos}
\subsection{Função de ativação Adam}
\subsection{Overfitting}
\subsection{Dropout}

\section{Metodologia}

A projeto \textit{CAPTCHA}, foi construído com base em uma arquitetura de redes neurais em \textit{Deep Learning} com cinco camadas convolucionais, sendo uma entrada e uma saída acrescido de três \textit{hidden layers}, mais especificamente Conv2D com os parâmetros de \textit{padding = ``same''} e ativação ``relu''. As próximas subseções descreverão os procedimentos que compõem o projeto, que são o dataset, instanciação, processamento, reconhecimento e quebra das imagens alfanuméricas CAPTCHA. O processo inclui também o pré-processamento de dados de entrada, estruturação da rede, função de predição e execução.

\subsection{Dataset}

O Dataset é montado utilizando uma funcionalidade desenvolvida pela equipe chamado \textbf{Gerador\_do\_CAPTCHA.py}. Sua tarefa é, como o nome sugere, gerar aleatóriamente imagens \textit{CAPTCHA} alfanuméricas. Para isso ele se vale das seguintes bibliotecas:

\begin{itemize}
    \item Datetime: coletar data e hora;
    \item CAPTCHA: biblioteca CAPTCHA que traz funções para criação de CAPTCHA de áudio e imagem (fonte: https://pypi.org/project/CAPTCHA/);
    \item Matplotlib: para gerar gráficos de avaliação da solução;
    \item Random: gerador de valores aleatórios do Python;
    \item String: manipulador de strings;
    \item OS: biblioteca responsável por chamadas ao Sistema Operacional;
\end{itemize}

O gerador de datasets elenca, inicialmente, os símbolos alfa-numéricos para criação do CAPTCHA, unindo as letras do alfabeto em baixa e alta caixa com os dígitos indo-arábicos (veja o trecho \ref{inicio})

\lstinputlisting[language=Python, label=inicio, caption=Instanciação de Dataset,xleftmargin=.05\textwidth, xrightmargin=.03\textwidth
]{Relatorio/trechos_de_codigo/instancia_dataset.py}

O texto do CAPTCHA é gerado por um string de tamanho fixo de cinco caracteres, escolhidos aleatoriamente dos símbolos disponíveis. A lib \textit{captcha} do python é aplicada para unir a string com a imagem, e para a geração da última. A lib recebe por parâmetro o texto e que é formatado para uma fonte específica e adicionados ruídos de traço e pontos (trecho \ref{string}). A adição desses últimos elementos é para deixar mais verossímil, já que, em tese, o CAPTCHA deve ser resolvido por um humano, como forma de segurança (Figura \ref{fig:captcha}).

\lstinputlisting[language=Python, label=string, caption=Montagem do CAPTCHA,xleftmargin=.05\textwidth, xrightmargin=.03\textwidth
]{Relatorio/trechos_de_codigo/monta_captcha.py}

\begin{figure}
    \centering
    \includegraphics[scale=0.3]{Relatorio/figuras/captcha.png}
    \caption{Exemplo de CAPTCHA gerado}
    \label{fig:captcha}
\end{figure}

É necessário uma quantidade suficientemente boa de imagens no dataset em função de dois parâmetros específicos: treinamento da rede neural e recursos computacionais. Dessa forma, foi instanciado um número de 1000 imagens na composição do dataset. Para que não se perca nenhum dataset anteriormente gerado, a cada nova execução cria-se uma pasta com ``\textit{nome}'' + ``\textit{data e hora de criação}''. Isso é para o caso de ser necessário executar com um determinado dataset especificamente (trecho \ref{dataset}). 

\lstinputlisting[language=Python, label=dataset, caption=Montagem do Dataset,xleftmargin=.05\textwidth, xrightmargin=.03\textwidth
]{Relatorio/trechos_de_codigo/gera_dataset.py}

\subsection{Estruturação da rede neural}

A Rede Neural recebe como parâmetro de entrada o número de símbolos com os quais ele irá trabalhar (trecho \ref{inicio})


\subsection{Instanciação}
Começo do código até o def preprocessing

\subsection{Pré-Processamento}

Descrição do pré-processamento

\subsection{Função de predição}

\subsection{Execução}

\section{Testes}
Descrição em detalhes dos testes que foram realizados.
Falar sobre a arquitetura que rodamos etc.

1. Vamos analisar Loss e Accuracy, da seguinte forma:
    - Em função da ativação (relu, sigmod etc)
    - Em função dos otimizadores (Adam, SGD etc)
    - Em função de cada dígito (inspirado no artigo, referência)
    - Em função dos datasets (inspirado no artigo, referência)
    - Alterações nas camadas da rede neural
    - Alterar a própria rede neural em si
    - Para cada teste, pegar a quantidade de epochs durante o ajuste
2. Análise de performance, como padrão aqui

\subsection{Resultados}

Gráficos e discussão sobre. Podemos pegar o artigo \cite{noury2020deep} e comparar com os trabalhos que estão descritos lá e ver o que a nossa solução tem de melhor.


%https://2captcha.com/?utm_source=google_ads&utm_medium=1001729&utm_campaign=world_search&utm_content=1&utm_term=%2Bcaptcha%20%2Bsolver&pos=&gclid=Cj0KCQjw6uT4BRD5ARIsADwJQ1_Zi-LR0uSCD_fXK88xb8V2rRKY9pEQYGJZw3x7MrGRUsreUA9mw84aAvYCEALw_wcB
\section{Conclusões}

Conclusões gerais do trabalho

%******************************************************************************
% Referências - Definidas no arquivo Relatorio.bib


\bibliographystyle{IEEEtran}

\bibliography{Relatorio}


%******************************************************************************

\vspace{20ex}

\vspace{3ex}

\end{document}