%% Adaptado de 
%% http://www.ctan.org/tex-archive/macros/latex/contrib/IEEEtran/
%% Traduzido para o congresso de IC da USP
%%*****************************************************************************
% Não modificar

\documentclass[twoside,conference,a4paper]{IEEEtran}

%******************************************************************************
% Não modificar
\usepackage{IEEEtsup} % Definições complementares e modificações.
\usepackage[latin1,utf8]{inputenc} % Disponibiliza acentos.
%\usepackage[english,brazil]{babel}
\usepackage[brazil,brazilian]{babel}
%% Disponibiliza Inglês e Português do Brasil.
\usepackage{latexsym,amsfonts,amssymb} % Disponibiliza fontes adicionais.
\usepackage{theorem} 
\usepackage[cmex10]{amsmath} % Pacote matemático básico 
\usepackage{url} 
%\usepackage[portuges,brazil,english]{babel}
\usepackage{graphicx}
\usepackage{amsmath}
\usepackage{amssymb}
\usepackage{color}
\usepackage[pagebackref=true,breaklinks=true,letterpaper=true,colorlinks,bookmarks=false]{hyperref}
\usepackage[tight,footnotesize]{subfigure} 
\usepackage[noadjust]{cite} % Disponibiliza melhorias em citações.
\usepackage{algorithmic}
\usepackage{algpseudocode} 
\usepackage{algorithm}
\usepackage{listings}
\usepackage{xcolor}

\definecolor{codegreen}{rgb}{0,0.6,0}
\definecolor{codegray}{rgb}{0.5,0.5,0.5}
\definecolor{codepurple}{rgb}{0.58,0,0.82}
\definecolor{backcolour}{rgb}{255,255,255}

\lstdefinestyle{mystyle}{
    backgroundcolor=\color{backcolour},   
    commentstyle=\color{codegreen},
    keywordstyle=\color{magenta},
    numberstyle=\tiny\color{codegray},
    stringstyle=\color{codepurple},
    basicstyle=\ttfamily\footnotesize,
    breakatwhitespace=false,         
    breaklines=true,                 
    captionpos=b,                    
    keepspaces=true,                 
    numbers=left,                    
    numbersep=5pt,                  
    showspaces=false,                
    showstringspaces=false,
    showtabs=false,                  
    tabsize=2
}

\lstset{style=mystyle}
%%*****************************************************************************

\begin{document}
\selectlanguage{brazil}
\renewcommand{\IEEEkeywordsname}{Palavras-chave}

%%*****************************************************************************

\urlstyle{tt}
% Indicar o nome do autor e o curso/nível (grad-mestrado-doutorado-especial)
\title{Project 3 - XX - AI}
\author{%
 \IEEEauthorblockN{Christian Maekawa\,\IEEEauthorrefmark{1}}
 \IEEEauthorblockA{\IEEEauthorrefmark{1}%
                   RA: XXX \\
                   E-mail: XXX@ic.unicamp.br}
 \IEEEauthorblockN{Giovane de Morais\,\IEEEauthorrefmark{1}}
 \IEEEauthorblockA{\IEEEauthorrefmark{1}%
                   RA: XXX \\
                   E-mail: g192683@dac.unicamp.br}
 \IEEEauthorblockN{Maísa Silva\,\IEEEauthorrefmark{1}}
 \IEEEauthorblockA{\IEEEauthorrefmark{1}%
                   RA: XXX \\
                   E-mail: XXX@ic.unicamp.br}
 \IEEEauthorblockN{Matteus Vargas\,\IEEEauthorrefmark{1}}
 \IEEEauthorblockA{\IEEEauthorrefmark{1}%
                   RA: XXX \\
                   E-mail: XXX@ic.unicamp.br}
 \IEEEauthorblockN{Stéfani Fernandes\,\IEEEauthorrefmark{1}}
 \IEEEauthorblockA{\IEEEauthorrefmark{1}%
                   RA: 147939 \\
                   E-mail: s147939@ic.unicamp.br}
}

%%*****************************************************************************

\maketitle

%%*****************************************************************************
% Resumo do trabalho
\begin{abstract}

\end{abstract}

% Indique três palavras-chave que descrevem o trabalho
\begin{IEEEkeywords}
 Palavras-chave
\end{IEEEkeywords}

%%*****************************************************************************
% Modifique as seções de acordo com o seu projeto

\section{Introdução}

%O presente relatório apresenta a descrição das estratégias adotadas para a resolução do Projeto 3 da disciplina Introdução à Inteligência Artificial, ministrada no primeiro semestre de 2020 pela professora Dra. Esther Luna Colombini.
Aqui deve ser apresentada a problemática que iremos abordar, porque ela é relevante e os conceitos que iremos abranger.

\section{Referencial teórico}
Aprendizado supervisionado
Regressão
entender que problema é de classificação
problema supervisionado
metricas são de classificação
Mais especificamente resolvendo problema de visão computacional
etc
\subsection{Ferramentas}

Vamos detalhar aqui a linguagem de programação, as bibliotecas e parâmetros que iremos utilizar.
Exemplo: linguagem Python, tensorflow, keras, relu, sigmoid, conv2d, numpy, matplotlib etc.
Explicar essas ferramentas aliviará e muito na hora de escrevermos a metodologia.

\section{Metodologia}

A projeto \textit{CAPTCHA}, foi construído com base em uma arquitetura de redes neurais em \textit{Deep Learning} com cinco camadas convolucionais, sendo uma entrada e uma saída acrescido de três \textit{hidden layers}, mais especificamente Conv2D com os parâmetros de \textit{padding = ``same''} e ativação ``relu''. As próximas subseções descreverão os procedimentos que compõem o projeto, que são o dataset, instanciação, processamento, reconhecimento e quebra das imagens alfanuméricas CAPTCHA. O processo inclui também o pré-processamento de dados de entrada, estruturação da rede, função de predição e execução.

\subsection{Dataset}

O Dataset é montado utilizando uma funcionalidade desenvolvida pela equipe chamado \textbf{Gerador\_do\_CAPTCHA.py}. Sua tarefa é, como o nome sugere, gerar aleatóriamente imagens \textit{CAPTCHA} alfanuméricas. Para isso ele se vale das seguintes bibliotecas:

\begin{itemize}
    \item Datetime: coletar data e hora;
    \item CAPTCHA: biblioteca CAPTCHA que traz funções para criação de CAPTCHA de áudio e imagem (fonte: https://pypi.org/project/CAPTCHA/);
    \item Matplotlib: para gerar gráficos de avaliação da solução;
    \item Random: gerador de valores aleatórios do Python;
    \item String: manipulador de strings;
    \item OS: biblioteca responsável por chamadas ao Sistema Operacional;
\end{itemize}

O gerador de datasets elenca, inicialmente, os símbolos alfa-numéricos para criação do CAPTCHA, unindo as letras do alfabeto em baixa e alta caixa com os dígitos indo-arábicos (veja o trecho \ref{inicio})

\lstinputlisting[language=Python, label=inicio, caption=Instanciação de Dataset,xleftmargin=.05\textwidth, xrightmargin=.03\textwidth
]{Relatorio/trechos_de_codigo/instancia_dataset.py}

O texto do CAPTCHA é gerado por um string de tamanho fixo de cinco caracteres, escolhidos aleatoriamente dos símbolos disponíveis. A lib \textit{captcha} do python é aplicada para unir a string com a imagem, e para a geração da última. A lib recebe por parâmetro o texto e que é formatado para uma fonte específica e adicionados ruídos de traço e pontos (trecho \ref{string}). A adição desses últimos elementos é para deixar mais verossímil, já que, em tese, o CAPTCHA deve ser resolvido por um humano, como forma de segurança (Figura \ref{fig:captcha}).

\lstinputlisting[language=Python, label=string, caption=Montagem do CAPTCHA,xleftmargin=.05\textwidth, xrightmargin=.03\textwidth
]{Relatorio/trechos_de_codigo/monta_captcha.py}

\begin{figure}
    \centering
    \includegraphics[scale=0.3]{Relatorio/figuras/captcha.png}
    \caption{Exemplo de CAPTCHA gerado}
    \label{fig:captcha}
\end{figure}

É necessário uma quantidade suficientemente boa de imagens no dataset em função de dois parâmetros específicos: treinamento da rede neural e recursos computacionais. Dessa forma, foi instanciado um número de 1000 imagens na composição do dataset. Para que não se perca nenhum dataset anteriormente gerado, a cada nova execução cria-se uma pasta com ``\textit{nome}'' + ``\textit{data e hora de criação}''. Isso é para o caso de ser necessário executar com um determinado dataset especificamente (trecho \ref{dataset}). 

\lstinputlisting[language=Python, label=dataset, caption=Montagem do Dataset,xleftmargin=.05\textwidth, xrightmargin=.03\textwidth
]{Relatorio/trechos_de_codigo/gera_dataset.py}

\subsection{Estruturação da rede neural}

A Rede Neural recebe como parâmetro de entrada o número de símbolos com os quais ele irá trabalhar (trecho \ref{inicio})


\subsection{Instanciação}
Começo do código até o def preprocessing

\subsection{Pré-Processamento}

Descrição do pré-processamento

\subsection{Função de predição}

\subsection{Execução}




% \begin{figure}
%     \centering
%     \includegraphics{ModeloRelatorio/figuras/3ndxd.png}
%     \caption{Caption}
%     \label{fig:captcha_exemplo}
% \end{figure}

% \begin{algorithm}{}
% \caption{Reconhecimento de Captcha}
% \label{alg:captcha}
%     \begin{algorithmic}[1]
%     \State simbolos $\leftarrow$ todos as letras do alfabeto em caixa baixa juntamente com os dígitos indo-arábicos
%     \STATE num\_symbols $\leftarrow$ numero total de símbolos
%     \State X,y $\leftarrow$ pré-processamento de dados (normalização dos dados, juntamente com outros ajustes pertinentes)
%     \State X\_train, y\_train $\leftarrow$ X[:970], y[:, :970] (divide parte do dataset em treinamento)
%     \State X\_test, y\_test $\leftarrow$ X[970:], y[:, 970:] (divide parte do dataset em teste)
    
%     \State net $\leftarrow$ cria rede neural (por padrão, rede de cinco camadas, sendo um entrada, uma saída e o restante \textit{hidden}, ativada por ``relu'' e otimizada por ``rmsprop'')
%     \State history $\leftarrow$ net.fit (ajuste de rede usando dataset de treino)
%     \State net.evaluate(dataset de teste)

%     \WHILE{sistema\_operacional.le\_imagens}
%         \State realiza a predicao
%     \ENDWHILE
%     \end{algorithmic}
% \end{algorithm}

% O primeiro a ser discutido é o Keras. Keras é uma API de Deep Learning escrita em Python, rodando sobre a plataforma de Machine Learning TensorFlow. Foi desenvolvido com o objetivo de permitir experimentação rápida. Ser capaz de passar da ideia para o resultado o mais rápido possível é essencial para fazer uma boa pesquisa (fonte: https://keras.io/about/).

% São importadas bibliotecas do python também como o "os", que serve para fazer chamadas no sistema operacional, o cv2 que é uma versão do OpenCV (Open Source Computer Vision Library) que, por sua vez, é uma biblioteca de software de visão computacional e Machine Learning de código aberto. O OpenCV foi desenvolvido para fornecer uma infraestrutura comum para aplicativos de visão computacional e acelerar o uso da percepção da máquina nos produtos comerciais (fonte: https://opencv.org/about/). 

% A "string" que, como o nome indica, serve para manipulação de strings.
% Por último, duas bibliotecas muito importantes para manipulação de dados e em algoritmos de Machine Learning: o numpy e o matplotlib. O NumPy é o pacote fundamental para a computação científica em Python. É uma biblioteca Python que fornece um objeto de matriz multidimensional, vários objetos derivados (como matrizes e matrizes mascaradas) e uma variedade de rotinas para operações rápidas em matrizes, incluindo matemática, lógica, manipulação de formas, classificação, seleção, I/O , transformadas discretas de Fourier, álgebra linear básica, operações estatísticas básicas, simulação aleatória entre outras operações (fonte: https://numpy.org/doc/stable/user/whatisnumpy.html).

% Já o Matplotlib é uma biblioteca de visualização em Python para gráficos 2D de matrizes. é multiplataforma construída em matrizes NumPy e projetada para trabalhar com a pilha SciPy mais ampla. Ele nos permite acesso visual a grandes quantidades de dados em imagens, consistindo em vários gráficos, como linha, barra, dispersão, histograma etc .
\section{Testes}
Descrição em detalhes dos testes que foram realizados.
Falar sobre a arquitetura que rodamos etc.

1. Vamos analisar Loss e Accuracy, da seguinte forma:
    - Em função da ativação (relu, sigmod etc)
    - Em função dos otimizadores (Adam, SGD etc)
    - Em função de cada dígito (inspirado no artigo, referência)
    - Em função dos datasets (inspirado no artigo, referência)
    - Alterações nas camadas da rede neural
    - Alterar a própria rede neural em si
    - Para cada teste, pegar a quantidade de epochs durante o ajuste
2. Análise de performance, como padrão aqui

\subsection{Resultados}

Gráficos e discussão sobre. Podemos pegar o artigo \cite{noury2020deep} e comparar com os trabalhos que estão descritos lá e ver o que a nossa solução tem de melhor.


%https://2captcha.com/?utm_source=google_ads&utm_medium=1001729&utm_campaign=world_search&utm_content=1&utm_term=%2Bcaptcha%20%2Bsolver&pos=&gclid=Cj0KCQjw6uT4BRD5ARIsADwJQ1_Zi-LR0uSCD_fXK88xb8V2rRKY9pEQYGJZw3x7MrGRUsreUA9mw84aAvYCEALw_wcB
\section{Conclusões}

Conclusões gerais do trabalho

%******************************************************************************
% Referências - Definidas no arquivo Relatorio.bib


\bibliographystyle{IEEEtran}

\bibliography{Relatorio}


%******************************************************************************

\vspace{20ex}

\vspace{3ex}

\end{document}