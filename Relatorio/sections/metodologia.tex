\section{Metodologia}

A projeto \textit{CAPTCHA}, foi construído com base em uma arquitetura de redes neurais em \textit{Deep Learning} com cinco camadas convolucionais, sendo uma entrada e uma saída acrescido de três \textit{hidden layers}, mais especificamente Conv2D com os parâmetros de \textit{padding = ``same''} e ativação ``relu''. As próximas subseções descreverão os procedimentos que compõem o projeto, que são o dataset, instanciação, processamento, reconhecimento e quebra das imagens alfanuméricas CAPTCHA. O processo inclui também o pré-processamento de dados de entrada, estruturação da rede, função de predição e execução.

\subsection{Dataset}

O Dataset é montado utilizando uma funcionalidade desenvolvida pela equipe chamado \textbf{Gerador\_do\_CAPTCHA.py}. Sua tarefa é, como o nome sugere, gerar aleatóriamente imagens \textit{CAPTCHA} alfanuméricas. Para isso ele se vale das seguintes bibliotecas:

\begin{itemize}
    \item Datetime: coletar data e hora;
    \item CAPTCHA: biblioteca CAPTCHA que traz funções para criação de CAPTCHA de áudio e imagem (fonte: https://pypi.org/project/CAPTCHA/);
    \item Matplotlib: para gerar gráficos de avaliação da solução;
    \item Random: gerador de valores aleatórios do Python;
    \item String: manipulador de strings;
    \item OS: biblioteca responsável por chamadas ao Sistema Operacional;
\end{itemize}

O gerador de datasets elenca, inicialmente, os símbolos alfa-numéricos para criação do CAPTCHA, unindo as letras do alfabeto em baixa e alta caixa com os dígitos indo-arábicos (veja o trecho \ref{inicio})

\lstinputlisting[language=Python, label=inicio, caption=Instanciação de Dataset,xleftmargin=.05\textwidth, xrightmargin=.03\textwidth
]{Relatorio/trechos_de_codigo/instancia_dataset.py}

O texto do CAPTCHA é gerado por um string de tamanho fixo de cinco caracteres, escolhidos aleatoriamente dos símbolos disponíveis. A lib \textit{captcha} do python é aplicada para unir a string com a imagem, e para a geração da última. A lib recebe por parâmetro o texto e que é formatado para uma fonte específica e adicionados ruídos de traço e pontos (trecho \ref{string}). A adição desses últimos elementos é para deixar mais verossímil, já que, em tese, o CAPTCHA deve ser resolvido por um humano, como forma de segurança (Figura \ref{fig:captcha}).

\lstinputlisting[language=Python, label=string, caption=Montagem do CAPTCHA,xleftmargin=.05\textwidth, xrightmargin=.03\textwidth
]{Relatorio/trechos_de_codigo/monta_captcha.py}

\begin{figure}
    \centering
    \includegraphics[scale=0.3]{Relatorio/figuras/captcha.png}
    \caption{Exemplo de CAPTCHA gerado}
    \label{fig:captcha}
\end{figure}

É necessário uma quantidade suficientemente boa de imagens no dataset em função de dois parâmetros específicos: treinamento da rede neural e recursos computacionais. Dessa forma, foi instanciado um número de 1000 imagens na composição do dataset. Para que não se perca nenhum dataset anteriormente gerado, a cada nova execução cria-se uma pasta com ``\textit{nome}'' + ``\textit{data e hora de criação}''. Isso é para o caso de ser necessário executar com um determinado dataset especificamente (trecho \ref{dataset}). 

\lstinputlisting[language=Python, label=dataset, caption=Montagem do Dataset,xleftmargin=.05\textwidth, xrightmargin=.03\textwidth
]{Relatorio/trechos_de_codigo/gera_dataset.py}

\subsection{Estruturação da rede neural}

A Rede Neural recebe como parâmetro de entrada o número de símbolos com os quais ele irá trabalhar (trecho \ref{inicio})


\subsection{Instanciação}
Começo do código até o def preprocessing

\subsection{Pré-Processamento}

Descrição do pré-processamento

\subsection{Função de predição}

\subsection{Execução}
